\begin{lstlisting}[style=PYTHON,frame=single,
 caption=Zum einlesen der Mundharmonika Schwingung
 captionpos=b,
 label=lst:read] 
 import numpy as np
 from TekTDS2000 import *
 
 scope = TekTDS2000()
 
 input()
 scope.saveCsv(filename='mic1000Hz.csv', ch=2)
 scope.saveCsv(filename='sig1000Hz.csv', ch=1)
 scope.plot(filename='mic1000Hz.png', ch=2)
 scope.plot(filename='sig1000Hz.png', ch=1)
 scope.getSamplingInterval()
 scope.getRecordLength()
 # ausgabe bei nachmessung:
 #TekTDS2000: Sampling interval: 2e-06 s
 #TekTDS2000: Record length: 2500
 \end{lstlisting}
 
  
  \begin{lstlisting}[style=PYTHON,frame=single,
   caption=Zum umwandeln der Mundharmonika Schwingung
   captionpos=b,
   label=lst:v1d]
   import numpy as np
   import matplotlib.pyplot as plt
   
   #Periodendauer
   def periodendauer():
       input_data = np.genfromtxt('tone.csv', delimiter=',')
       my_data = input_data[:,1:2];
       
       plt.figure(figsize=(10, 5), dpi=800)
       plt.xticks(range(0,271,10))
       plt.plot(my_data[0:270], label='Signal')
       plt.plot((44, 44), (-0.008, 0.008), 'k-', color='red')      # anfang Periode
       plt.plot((153, 153), (-0.008, 0.008), 'k-', color='red')    # ende Periode
       plt.ylabel("Amplitude")
       plt.xlabel("Abtastzeitpunkt")
       plt.text(57, -0.0065, '$periodendauer = 109 * 10^{-5} s$', fontsize=14)
       plt.legend()
       plt.show()
   
   
   # Fouriertransformation
   def mach_fft(data):
       fft = np.abs(np.fft.fft(data, axis=0))
       plt.figure(figsize=(10, 5), dpi=800)
       plt.xlabel("Frequenz [$Hz$]")
       plt.ylabel("Amplitude")
       plt.plot(range(0,16000,40), fft[:400], label="Fourierdom�ne")   # gek�rzt bei 400, rest uninteressant
       plt.plot((910, 910), (3, 0), 'k-', color='red',label="Grundschwingung = 910 Hz") # Grundfrequenz
       #plt.plot((1820, 1820), (3, 0), 'k-', color='red') # 1 Harmonische
       #plt.plot((2730, 2730), (3, 0), 'k-', color='red') # 2 Harmonische
       plt.legend()
       plt.show()
   \end{lstlisting}