%
% CHAPTER Versuch 1
%
\chapter{Bestimmung der Tonhöhe eines akustischen Signals}
\label{chap:VERSUCH_1}
Bestimmen der Frequenz eines Tones und Fouriertransformation.
 
\section{Fragestellung, Messprinzip, Aufbau, Messmittel}
\label{chap:VERSUCH_1_FRAGESTELLUNG}
In diesem ersten Versuch soll die Tonhöhe eines akustischen Signals bestimmt werden. Als akustischer Signalgeber wird eine Munharmonika verwendet. Da bei einer Mundharmonika Töne durch einzelne dicht aneinanderliegende Luftkanäle erzeugt werden, ist es für einen Laien auf Anhieb nicht möglich einen einzelnen Ton zu spielen. Damit nun also genau ein Ton zu hören ist wurden alle Luftkanäle bis auf einen abgeklebt.
Der Ton der Mundharmonika wird dann mit einem Mikrophon aufgenommen. Dieses ist an ein Oszilloskop angeschlossen. 
Da die Signale in analoger Form vorliegen, muss für die weitere Verarbeitung des Signals eine Digitalisierung vorgenommen werden.
Dabei ist das Abtastintervall $\Delta t$ gerade der zeitliche Abstand zwischen zwei Messungen. Die Abtastfrequenz ist dann gerade der Kehrwert zum Abtastinterval $\frac{1}{\Delta t}$. Diese Werte können im Unterabschnitt Messwerte in der Tabelle \ref{tab:Eigenschaften} entnommen werden. Die Digitalisierung erfolgt durch das Oszilloskop.
\\
Um dann einen Signalausschnitt aufzunehmen wurde der "\textit{Single Sequence"} Modus im Oszilloskop eingestellt. Mit einem Pythonskript wurden dann 2500 Messwerte vom Oszilloskop ausgelesen.
Aus diesem Signal im Zeitbereich wird dann das Spektrum mithilfe der Fouriertransformation berechnet. Die Umrechnung in das Amplitudenspektrum
erfolgt mit Hilfe einer Numpy Funktion, die auf der X-Achse jedoch nicht die Frequenz aufträgt, sondern die Einheit \textit{Anzahl Schwingungen innerhalb der gesamten Signaldauer}. Interessant ist jedoch die Frequenz. Deshalb muss diese noch mit folgender Formel errechnet werden.
\begin{equation}
f = \dfrac{n}{M \cdot \Delta t}
\end{equation}




\section{Messwerte}
\label{chap:VERSUCH_1_MESSWERTE}


\begin{table}[H]
\centering
\begin{tabular}{l|r}
Eigenschaft & Wert \\ \hline
Grundfrequenz in Hz & 917.43\\
Grundperiode in ms & 1.09\\
Abtastfreq in kHz & 100\\
Signaldauer in s & 0.025\\
Abtastinterval $\Delta t$ in s & $1 \cdot 10^{-5}s$\\
Signalänge $M$ Abtastungen & 2500\\
\end{tabular}
\caption{Eigenschaften der Messung und Grundfrequenz des Tones}
\label{tab:Eigenschaften}
\end{table}

\begin{figure}[H]
\centering
\includegraphics[width=170mm]{tone.png}
\caption{Signal im Zeitbereich}
\label{img:SIGNALZEITBEREICH}
\end{figure}

\begin{figure}[H]
\centering
\includegraphics[width=165mm]{periodendauer.png}
\caption{Ausschnitt Signal im Zeitbereich mit eingezeichneter Periode}
\label{img:SIGNALPERIODE}
\end{figure}

\section{Auswertung und Interpretation}
\label{chap:VERSUCH_1_AUSWERTUNG}
In Bild \ref{img:SIGNALZEITBEREICH} ist das Signal der Mundharmonika im Zeitbereich zu sehen. Auf den ersten Blick scheint es sich wie erwartet um ein periodisches Signal zu handeln. Auf der X-Achse ist die Zeit in Sekunden und auf der Y-Achse die Amplitude in Volt aufgetragen. Nun soll die Grundfrequenz, sowie die Grundperiodendauer aus dem Signal im Zeitbereich bestimmt werden. 
