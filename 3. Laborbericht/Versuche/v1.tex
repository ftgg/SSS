%
% CHAPTER Versuch 1
%
\chapter{Bestimmung der Tonhöhe eines akustischen Signals}
\label{chap:VERSUCH_1}
In diesem ersten Versuch soll die Tonhöhe eines akustischen Signals bestimmt werden. Als akustischer Signalgeber soll eine Mundharmonika dienen. Da bei einer Mundharmonika Töne durch einzelne Luftkanäle erzeugt werden, die dicht aneinander liegen, ist es für einen Laien auf Anhieb nicht möglich einen einzelnen Ton zu spielen. Damit nun also genau ein Ton zu hören ist wurden alle Luftkanäle bis auf einen abgeklebt.
Den Ton der Mundharmonika wird dann mit einem Mikrophon aufgenommen, welches an ein Oszilloskop angeschlossen ist. Um dann einen Signalausschnitt auzunehmen wurde der "\textit{Single Sequence"} Modus benutzt. Mit einem Pythonskript wurden dann 2500 Messwerte vom Oszilloskop ausgelesen.
 
\section{Fragestellung, Messprinzip, Aufbau, Messmittel}
\label{chap:VERSUCH_1_FRAGESTELLUNG}


\section{Auswertung und Interpretation}
\label{chap:VERSUCH_1_AUSWERTUNG}
