%
% CHAPTER Versuch 2
%
\chapter{Frequenzgang von Lautsprechern}
Es gilt für zwei Lautsprecher die Amplitude und Phasenverschiebung eines akustischen Signales festzustellen.
Hierzu wird das Eingangssignal des Lautsprechers, welches in einem Sinusgenerator erzeugt wird, auf dem Oszilloskop dargestellt. Und das Ausgangssignal mittelt eines Mikrophons aufgenommen ebenfalls auf dem Oszilloskop angezeigt.

\label{chap:Aufnahme_eines_Dunkelbildes}

\section{Fragestellung, Messprinzip, Aufbau, Messmittel}
Für diesen Versuch werden 17 Messungen mit unterschiedlichen Frequenzen pro Lautsprecher durchgeführt.
Das erzeugte Sinussignal gilt als Referenzwert welcher bei jeder Messung die Amplitude von 1,5V besitzt.
Das zu vergleichende Ausgangssignal des Lautsprechers wird über ein Mikrophon aufgenommen, welches einen stets gleichen Abstand zum Lautsprecher hat.
Beim vergleichen beider Werte sollte sowohl eine Phasenverschiebung als auch ein Amplitudenunterschied festgestellt werden. dazu Später mehr.
Die einzelnen Messungen werden mit Frequenzen: $f = 100,200,300,400,500,700,850 Hz, 1, 1.2, 1.5, 1.7, 2, 3, 4, 5, 6, 10kHz$ durchgeführt. Hierbei ist zu beachten, dass die Spannung des Sinusgenerators bei jeder Messung auf 1.5V gestellt werden, da dieser mit jeder Messung ein wenig variiert.
Die Messungen werden im "Single Sequence Mode" durchgeführt, was die Lärmbelastung verringert.
Bei jeder Messung wird sowohl die Amplitude der Ausgangssignals als auch die Phasenverschiebung zur Eingangsspannung bemessen und protokolliert \ref{} TODO
\label{chap:VERSUCH_2_FRAGESTELLUNG}

\section{Messwerte}
\begin{table}[H]
\centering
\begin{tabular}{ccc}
  Frequenz in $Hz$ & Amplitude in $mV$ &  Phasenverschiebung in $°$ \\
  100 & 50.2 & $-3.60$ \\
  200 & 134.0 & $-578.88$ \\
  300 & 95.2 & $-547,20$ \\
  400 & 60.4 & $-504$ \\
  500 & 49.2 & $-493.2$ \\
  700 & 39.6 & $-490.03$ \\
  850 & 43.2 & $-482.40$ \\
  1000 & 42.0 & $-469.44$ \\
  1200 & 35.2 & $-460.22$ \\
  1500 & 33.6 & $-444.24$ \\
  1700 & 30.8 & $-437.11$ \\
  2000 & 32.0 & $-421.92$ \\
  3000 & 37.6 & $-414.00$ \\
  4000 & 46.0 & $-1054.008$ \\
  5000 & 26.8 & $-982.80$ \\
  6000 & 19.4 & $-974.88$ \\
  10000 & 15.4 & $-752.40$ \\
 \end{tabular}
\label{tab:MLg}
\caption{Messwerte Lautsprecher groß}
\end{table}

\begin{table}[H]
\centering
\begin{tabular}{ccc}
  Frequenz in $Hz$ & Amplitude in $mV$ &  Phasenverschiebung in $°$ \\
  100 &  6.6& $-50.40$ \\
  200 & 10.4 & $-380.16$ \\
  300 & 15.0 & $-409.68$ \\
  400 & 23.8 & $-436.32$ \\
  500 & 34.6 & $-442.80$ \\
  700 & 70.0 & $-501.12$ \\
  850 & 68.0 & $-574.20$ \\
  1000 & 50.0 & $-471.60$ \\
  1200 & 40.0 & $-463.68$ \\
  1500 & 52.8 & $-509.04$ \\
  1700 & 53.6 & $-445.68$ \\
  2000 & 51.2 & $-432.00$ \\
  3000 & 50.4 & $-394.56$ \\
  4000 & 52.0 & $-1028.16$ \\
  5000 & 12.8 & $-1022.4$ \\
  6000 & 8.2 & $-914.40$ \\
  10000 & 19.4 & $-727.20$ \\
 \end{tabular}
\label{tab:MLk}
\caption{Messwerte Lautsprecher klein}
\end{table}
\label{chap:VERSUCH_2_MESSWERTE}

\section{Auswertung und Interpretation}
