%---------------
%╔═╗╔═╗╔╦╗╦ ╦╔═╗
%╚═╗║╣  ║ ║ ║╠═╝
%╚═╝╚═╝ ╩ ╚═╝╩  
%---------------
\documentclass[12pt,oneside,a4paper]{report}

% DOCUMENT SETUP
\usepackage[left=3cm, 
			right=2.5cm, 
			top=2.5cm, 
			bottom=2.5cm, 
			includehead, 
			includefoot]{geometry}

% line spacing
\usepackage{setspace}
\setstretch{1,25} % 15/12 --> 1.25

%de­fines Adobe Times Ro­man as de­fault text font
\usepackage{mathptmx}
\usepackage{times} % needed for acronym package

%PDF linking package
\usepackage[hidelinks]{hyperref}

% Language Setup
\usepackage[ngerman]{babel}
% language specific bibliography style
\usepackage[numbers]{natbib}
\usepackage[fixlanguage]{babelbib}
\selectbiblanguage{german}
% bliographystyle setup
% default style names: apalike alphadin ieeetr IEEEtranSN apalike2 alphadin 
% babel specific: babplain, babplai3, babalpha, babunsrt, bababbrv, bababbr3 unsrt 
\bibliographystyle{unsrturl}

% encoding setup
% T1 font encoding for languages that use a latin alphabet
\usepackage[T1]{fontenc} 

% enhanced input encoding handling - utf8 for äÄüÜöÖß...
\usepackage[utf8]{inputenc}
%\usepackage{ucs}%utf8x suppart

% after babel - set chapter string
\AtBeginDocument{\renewcommand{\chaptername}{}}

% enumeration
\usepackage{enumitem}
% tabular extension tabularx
\usepackage{tabularx}

% math packages
\usepackage{amsmath}
\usepackage{nicefrac}
\usepackage{amsthm}
\usepackage{amsbsy}
\usepackage{amssymb}
\usepackage{amsfonts}
\usepackage{MnSymbol}

% patches for latex
\usepackage{fixltx2e}

%special characters
\usepackage{amssymb}
\usepackage{upgreek,textgreek}

% acronym package
\usepackage[printonlyused, footnote]{acronym}

% breakable text in \seqsplit{}
\usepackage{seqsplit}

% \textmu
\usepackage{textcomp}

% package provides a way to compile sections of a document using the same preamble as the main document
\usepackage{subfiles}

% driver-independent color extension - used by listings,tabularx
\usepackage[usenames,dvipsnames,table,xcdraw]{xcolor}

% -- SYNTAX HIGHLIGHTING --
\usepackage{listings}
%\input{cfgs/listings/listings_def_lang_bash-cmd.tex} % adds style BASH_CMD
%\input{cfgs/listings/listings_def_lang_bash-script.tex} % adds style BASH_SCRIPT
\input{cfgs/listings/listings_def_lang_latex.tex} % adds style LATEX
%\input{cfgs/listings/listings_def_lang_matlab.tex} % adds style MATLAB
\input{cfgs/listings/listings_def_lang_python.tex} % adds style PYTHON
%\input{cfgs/listings/listings_def_lang_c++.tex} % adds style CPP
%\input{cfgs/listings/listings_def_lang_c.tex} % adds style C
%\input{cfgs/listings/listings_def_lang_json.tex} % adds style JSON

% HEADLINE CFG
\usepackage{fancyhdr} % Headers and footers
\usepackage{lastpage}
\usepackage{nopageno}
\setlength{\headheight}{1.5cm}
\pagestyle{fancy} % All pages have headers and footers
\fancyhead{} % Blank out the default header
\fancyfoot{} % Blank out the default footer
\fancyhead[L]{}
\fancyhead[C]{}
\fancyhead[R]{}
\fancyfoot[L]{}
\fancyfoot[C]{\thepage}
\fancyfoot[R]{}
% override plain page style for \part, \chapter or 
% \maketitle, which implicit specifies plain page style
\input{cfgs/fancyhdr/fancyhdr_pagestyle_plain.tex}
% set list pagestyle
\input{cfgs/fancyhdr/fancyhdr_pagestyle_lists.tex}

\renewcommand{\chaptermark}[1]{\markright{#1}{}}
\renewcommand{\sectionmark}[1]{\markright{#1}{}}
\renewcommand{\headrulewidth}{0pt}
\renewcommand{\footrulewidth}{0pt}

	
\usepackage{verbatim}
\usepackage{graphicx}
\usepackage{epstopdf}

% floating prevention packages
\usepackage{float}    % used with [H] positioning parameter
\usepackage{placeins} % \FloatBarrier 

% tikz packages
\usepackage{tikz}
\usepackage{caption}
\usepackage[list=true,listformat=simple]{subcaption}
\usepackage{standalone}
\usepackage{pgfplots}


% include only specified tex files - uncommend unneeded
\includeonly{preface/cover,
             preface/abstract,
             preface/tableofcontents,
             preface/listoffigures,
             preface/listoftables,
             preface/lstlistoflistings,
             appendix/bibliography}

%-------------------
%╔═╗╔╦╗╦═╗╦╔╗╔╔═╗╔═╗
%╚═╗ ║ ╠╦╝║║║║║ ╦╚═╗
%╚═╝ ╩ ╩╚═╩╝╚╝╚═╝╚═╝
%-------------------
\newcommand{\strLecture}{Signale, Systeme und Sensoren}
\newcommand{\strDate}{\today}
\newcommand{\strAuthorA}{Th. Gnädig}
\newcommand{\strAuthorB}{F. Gendusa}
\newcommand{\strAuthorAEmail}{thgnaedi@htwg-konstanz.de}
\newcommand{\strAuthorBEmail}{fagendus@htwg-konstanz.de}
% Versuchsbeschreibung 
\newcommand{\strTopic}{Kalibrierung von Digitalkameras}
\newcommand{\strAbstract}{In diesem Experiment befassen wir uns mit der Kalibrierung von digitalen Kameras.
Dabei finden Methoden Einsatz, wie sie der Art nach auch in der Industrie Einsatz finden.}
% hyperref customization
\hypersetup{
	pdftitle    ={\strTopic}, % title
	pdfsubject	={\strLecture}, % subject of the document
	pdfauthor	={\strAuthorA, \strAuthorB}, % author
	pdfkeywords	={}, % list of keywords
	pdfcreator	={}, % creator of the document
	pdfproducer	={}, % producer of the document
	colorlinks=false, % false: boxed links; true: colored links
	linkcolor=red, % color of internal links (change box color with linkbordercolor)
    citecolor=green, % color of links to bibliography
    filecolor=magenta, % color of file links
    urlcolor=cyan, % color of external links
	%bookmarks=true, % show bookmarks bar?
	unicode=true, % non-Latin characters in Acrobat’s bookmarks
	pdftoolbar=true, % show Acrobat’s toolbar?
	pdfmenubar=true, % show Acrobat’s menu?
    pdffitwindow=false, % window fit to page when opened
	pdfnewwindow=true % links in new PDF window
}

%-----------------------------------------
% ╔╗ ╔═╗╔═╗╦╔╗╔  ╔╦╗╔═╗╔═╗╦ ╦╔╦╗╔═╗╔╗╔╔╦╗ 
% ╠╩╗║╣ ║ ╦║║║║   ║║║ ║║  ║ ║║║║║╣ ║║║ ║  
% ╚═╝╚═╝╚═╝╩╝╚╝  ═╩╝╚═╝╚═╝╚═╝╩ ╩╚═╝╝╚╝ ╩  
%-----------------------------------------

\begin{document}
\pagenumbering{Roman} 

%\setcounter{section}{0}
\include{preface/cover}

\include{preface/abstract}
\clearpage

%
% TABLE OF CONTENTS
%
\include{preface/tableofcontents}

%
% Abbildungsverzeichnis
%
\include{preface/listoffigures}

%
% Tabellenverzeichnis
%
\include{preface/listoftables}

%
% Listingverzeichnis
%
\include{preface/lstlistoflistings}


%--------------------------
% ╔═╗╦ ╦╔═╗╔═╗╔╦╗╔═╗╦═╗╔═╗ 
% ║  ╠═╣╠═╣╠═╝ ║ ║╣ ╠╦╝╚═╗ 
% ╚═╝╩ ╩╩ ╩╩   ╩ ╚═╝╩╚═╚═╝ 
%--------------------------

\pagenumbering{arabic} 
\setcounter{page}{1}
%
% CHAPTER Einleitung
%
\chapter{Einleitung}
\label{chap:EINL}
Dieser zweite Versuch der Vorlesung Signale Systeme und Sensoren beschäftigt sich mit der Kalibrierung von digitalen Kameras. Aufgeteilt wird der Versuch in folgende vier Teile:
\begin{itemize}
\item{Aufnahme und Analyse eines Grauwertkeiles}
\item{Aufnahme eines Dunkelbildes}
\item{Aufnahme eines Weißbildes}
\item{Pixelfehler}
\end{itemize}
Wie alle Sensoren sind auch digitale Bildsensoren in der Praxis nicht völlig fehlerfrei. So unterliegen sie auch systematischen Fehlern und zeigen auch Rauschverhalten auf. Im Gegensatz zu einem herkömmlichen Sensor, der genau einen Wert zu einem Zeitpunkt misst \textit{(z.B. ein einfacher Temperatursensor)} besteht ein Bildsensor aus einer Matrix voll von Helligkeitssensoren. Dies sind die sogennanten Pixel. Aufgrund von Fertigungstolleranzen weisen nicht alle Pixel exakt das selbe Verhalten auf, so dass es selbst bei einer makelos homogenen Fläche, die abgelichtet wird, bei welcher alle Pixel den selben Wert haben müssten Differenzen zwischen den Pixeln gibt. So wird also streng genommen nicht ein Bildsensor kalibriert, sondern einzelne Helligkeitssensoren(Pixel) einer Matrix(Bildsensor). In diesem Fall handelt es sich um eine Logitech Webcam mit einer Auflösung von 640 x 480 Pixeln. Bei der Kalibrierung dieser Kamera wird sich in diesem Versuch nicht um die Korrektur fehlerbehafteter Farbwiedergaben gekümmert, sondern lediglich um die Helligkeitsinterpretation der einzelnen Pixel. So werden also alle aufgenommenen Bilder in Schwarz/Weiß umgewandelt und in dem verlustfreihen png Format abgespeichert. In folgenden Abschnitten werden Fehler, die zu Bildrauschen, bzw. Bildfehlern führen aufgeführt und teilweise versucht zu korrigieren.


%
% CHAPTER Versuch 1
%
%
% CHAPTER Versuch 1
%
\chapter{Aufnahme und Analyse eines Grauwertkeiles}
\label{chap:VERSUCH_1}

\section{Fragestellung, Messprinzip, Aufbau, Messmittel}
\label{chap:VERSUCH_1_FRAGESTELLUNG}


\section{Auswertung und Interpretation}
\label{chap:VERSUCH_1_AUSWERTUNG}


%
% CHAPTER Versuch 2
%
%
% CHAPTER Versuch 2
%
\chapter{Genauigkeit der AD-Wandlung}

\label{chap:VERSUCH_2}

\section{Fragestellung, Messprinzip, Aufbau, Messmittel}
\label{chap:VERSUCH_2_FRAGESTELLUNG}

\section{Messwerte}
\label{chap:VERSUCH_2_MESSWERTE}

\section{Auswertung und Interpretation}
\label{chap:AUSWERTUNGUNDINTERPRETATION}






%
% CHAPTER Versuch 3
%
%
% CHAPTER Versuch 3
%
\chapter{Versuch 3}
\label{chap:VERSUCH_3}

\section{Fragestellung, Messprinzip, Aufbau, Messmittel}
\label{chap:VERSUCH_3_FRAGESTELLUNG}

\section{Messwerte}
\label{chap:VERSUCH_3_MESSWERTE}

\section{Auswertung}
\label{chap:VERSUCH_3_AUSWERTUNG}

\section{Interpretation}
\label{chap:VERSUCH_3_INTERPRETATION}

%
% CHAPTER Versuch 4
%
%
% CHAPTER Versuch 4
%
\chapter{Pixelfehler}
Je nach Qualität des Bildsensors gibt es durch den Fertigungsprozess funktionsuntüchtige Pixel, welche das Bild verfälschen, diese gilt es in diesem Kapitel zu korrigieren.
\label{chap:Pixelfehler}

\section{Fragestellung, Messprinzip, Aufbau, Messmittel}
Um unser Testbild \ref{img:Grauwertkeil} zu korrigieren, gilt es nun den durch das Dunkelbild berechnete Offset der Pixel zu entfernen. Desweiteren wird das Bild noch durch das in V3 \ref{chap:VERSUCH_3} berechnete und normierte Weissbild geteilt um die Sensitivität anzupassen.
Im Anschluss wird uns ein Pythonskript TODO REF die differenz der einzelnen Bereiche zwischen dem Original Graukeilbild und dem korrigierten Graukeilbild berechnen. Hierzu wird das Bild zunächst in die einzelnen Graustufen zerlegt und im anschluss der durchschnittliche Farbwert dieser Graustufe wie auch die Standartabweichung, was dem rauschen der Pixel entspricht, berechnet. Durch vergleichen dieser Werte sollte sich eine Besserung des Bildes feststellen lassen. Die besserung sollte vorallem im beseitigen des rauschens liegen, was die Standartabweichung der einzelnen Keile sichtbar senken sollte.
\label{chap:VERSUCH_4_FRAGESTELLUNG}

\section{Messwerte}
\begin{figure}[H]
\centering
\includegraphics[width=150mm]{graustufen.png}
\caption{Markierter Offset}
\label{img:struckpixel.png}
\end{figure}
\label{chap:VERSUCH_4_MESSWERTE}

\section{Auswertung}
Bei genauerer Betrachtung des Kontrast maximierten Dunkelbildes \ref{img:struckpixel.png} fallen einige weisse Punkte ins Auge. Hierbei handelt es sich um hotpixel, welche durch längere Belichtungszeit in die Sättigung gehen oder um stuckpixel, welche sich stehts auf ihrem maximalwert bleiben $(rot markiert)$
\label{chap:VERSUCH_4_AUSWERTUNG}

\section{Interpretation}
\label{chap:VERSUCH_4_INTERPRETATION}

%
% CHAPTER Anhang
%
\renewcommand\thesection{A.\arabic{section}}
\renewcommand\thesubsection{\thesection.\arabic{subsection}}

\chapter*{Anhang}
\label{chap:APPENDIX}
\addcontentsline{toc}{chapter}{Anhang}
%\setcounter{chapter}{0}
\addtocounter{chapter}{1}
\setcounter{section}{0}

\section{Quellcode}
\label{chap:APPENDIX_SOURCECODE}

\subsection{Quellcode Versuch 1}
\label{chap:APPENDIX_SOURCECODE_V1}

\subsection{Quellcode Versuch 2}
\label{chap:APPENDIX_SOURCECODE_V2}

\subsection{Quellcode Versuch 3}
\label{chap:APPENDIX_SOURCECODE_V3}

\subsection{Quellcode Versuch 4}
\label{chap:APPENDIX_SOURCECODE_V4}


\section{Messergebnisse}
\label{chap:APPENDIX_MEASUREMENT_SOURCE}

%
% Literaturverzeichnis
%
\include{appendix/bibliography}

\end{document}
%------------------------------------
% ╔═╗╔╗╔╔╦╗  ╔╦╗╔═╗╔═╗╦ ╦╔╦╗╔═╗╔╗╔╔╦╗
% ║╣ ║║║ ║║   ║║║ ║║  ║ ║║║║║╣ ║║║ ║ 
% ╚═╝╝╚╝═╩╝  ═╩╝╚═╝╚═╝╚═╝╩ ╩╚═╝╝╚╝ ╩ 
%------------------------------------