%
% CHAPTER Versuch 2
%
\chapter{Aufnahme eines Dunkelbildes}
In diesem Versuch wird ein Dunkelbild zur Beseitigung des Offsets jedes Pixel erstellt.
\label{chap:Aufnahme_eines_Dunkelbildes}

\section{Fragestellung, Messprinzip, Aufbau, Messmittel}
Im zweiten Teil der Kalibrierung, werden zehn Dunkelbilder aufgenommen um den Pixel Offset zu bestimmen. Es werden mehrere Bilder benötigt um den Einfluss des thermischen Rauschens der Ausleseelektronik auf das Messergebnis möglichst gering zu halten. Das Pythonskript \ref{lst:Dunkelbild} berechnet den pixelweisen Mittelwert und speichert das Durchschnittsbild ab. Das so erstellte neue Bild, entspricht dem Mittelwert der zehn aufgenommenen Bilder und sollte jetzt weitgehendst nur noch den systematischen Fehler der jeweiligen Pixel, für die gewählte Belichtungszeit, enthalten. Der Offset in der Übertragungsfunktion des jeweiligen Pixels ist auf Fertigungstoleranzen und dem so entstandenen systematischen Fehler zurückzuführen. Das Mittelwertbild \ref{img:dunkelbild_kontrastmax} enthält jetzt nur noch minimales thermisches Rauschen und kann somit als Pixelweiser Offset für spätere Korrekturen verwendet werden.
\label{chap:VERSUCH_2_FRAGESTELLUNG}

\section{Messwerte}
\begin{figure}[H]
\centering
\includegraphics[width=150mm]{dunkelbildMean_maximiert.png}
\caption{Kontrastmaximiertes Dunkelbild}
\label{img:dunkelbild_kontrastmax}
\end{figure}
\label{chap:VERSUCH_2_MESSWERTE}

\section{Auswertung und Interpretation}
\label{chap:VERSUCH_2_AUSWERTUNG}
Bei genauerem Betrachten des kontrastmaximierten Dunkelbildes sind einige graue und weiße Pixel erkennbar. Diese sind aber lediglich durch die Kontrasmaximierung zu sehen. Dies sind also Pixel, die einen Offset haben, welche dann später noch mittels Dunkelbild korrigiert werden.
Das Dunkelbild entspricht sonst den Erwartungen, da die Mehrheit der Pixel schwarz ist $farbwert \approx 0$. Ist dieser Wert exakt null findet sich hier kein Dunkelstrom \textit{(Offset)} vor. Jedoch scheinen bei genauerer Betrachtung zu viele Pixel exakt bei Wert null zu sein. Vermutlich hat die Kamera eine integrierte Funktion, die schon versucht den Dunkelstrom zu kompensieren.
Das Dunkelbild kann jetzt zum korrigieren unseres Grauwertkeilbildes verwendet werden. Der Offset der Übertragungsfunktion eines Pixels wird durch Subtraktion des korespondierenden Pixels im Dunkelbild kompensiert. Das heißt also es wird das ganze Dunkebild vom Graustufenbild subtrahiert, um alle Pixel zu korrigieren.
