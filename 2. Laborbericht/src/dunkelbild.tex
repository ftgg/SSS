\begin{lstlisting}[style=PYTHON,frame=single,
 caption=Berechnen von dem Dunkelbild und Weißbild,
 captionpos=b,
 label=lst:Dunkelbild]
 
 import cv2read as cread
 import numpy as np
 import cv2

 def getimgmean(filename):							# berechnet das durchschnittsbild
     avg = np.zeros((480, 640), dtype=np.double)	# zu beginn ein Schwarzes Bild
     for i in range(1,11):							# alle 10 aufgenommenen Bilder mit gewichtung 1/10
 						# zu einem durchschnitts Bild zusammen addieren.
         avg = avg + cread.bildoeffnen("../Bilder/" + filename + str(i) + ".png") /10
     return avg
 
 bild = getimgmean("weissbild_")					# durchschnittliches Weissbild berechnen
 img = np.array(bild)								# in ein numpy array in uint8 umwandeln
 cv2.imwrite("../Bilder/weissbildMean.png", img)	# errechnetes Weissbild Speichern
 
 \end{lstlisting}