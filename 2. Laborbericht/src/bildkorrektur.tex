\begin{lstlisting}[style=PYTHON,frame=single,
 caption=Bildkorrektur,
 captionpos=b,
 label=lst:Bildkorrektur]
 import numpy as np
 import dunkelbild as dk
 import cv2read as cvread
 import matplotlib.pyplot as plt
 import cv2
 
 darkimg = dk.getimgmean("dunkelbild_")  # Dunkelbild durchschnitt
 whiteimg = dk.getimgmean("weissbild_")  # Weissbild durchschnitt
 wimg_cmp = whiteimg - darkimg           # Weissbild ohne thermisches Rauschen
 whiteNorm = wimg_cmp / np.mean(wimg_cmp)# Weissbild Normiert
 
 										# Bild zur korrektur einlesen
 img2 = np.array(cvread.bildoeffnen("../Bilder/graustufen.png")) # original Bild
 img = (img2 - darkimg) 				# PixelOffset abziehen
 for x in range(0,img.shape[1]):
     for y in range(0,img.shape[0]):
         if not whiteNorm[y][x] == 0:	# keine Division durch 0 zulassen
 	    								# Pixelweise durch Weissbild teilen, Sensitivität anpassen
             img[y][x] = img[y][x] / whiteNorm[y][x]
 
 cv2.imwrite("../Bilder/graustufen_korrigiert.png", img) # korrigiertes Bild
 \end{lstlisting}