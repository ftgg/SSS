\begin{lstlisting}[style=PYTHON,frame=single,
 caption=einteilung der Graustufen sowie die Berechnung derer Mittelwerte und StdAbw.
 captionpos=b,
 label=lst:Graustufen]
 import cv2read as cread
 import matplotlib.pyplot as plt
 import numpy as np

 def avg_stdabw(array):		# berechnet die Standartabweichung des Bildes
     avg = np.mean(array)   # Durchschnittswert des Bildes bestimmen
     std = 0
     for y in range(0,array.shape[0]):
         for x in range(0,array.shape[1]): # Standartabweichung bei jedem Pixel berechnen
             std = std + ((array[y][x] - avg) ** 2) / (array.size -1) # quadratischer Fehler
     std = np.sqrt(std)		# da std mit ^2 berechnet wurde, benötigt man jetzt noch die Wurzel
     return avg, std		# gibt durchschnittswert und Standartabweichung zurück

 def subimg(img):			# Teilt das Graustufenbild in die einzelnen Graustufen auf
 							# mindestens 10 pixel differenz zwischen den werten,
 							# da die übergänge nicht exakt gerade sind
     return npframe[:,0:110],npframe[:,120:260],npframe[:,280:410],npframe[:,420:560],npframe[0:460,570:640]
     
 frame = cread.bildoeffnen("../Bilder/graustufen_korrigiert.png")
 npframe = np.array(frame, dtype=np.uint8)
 subimgs = [] 				# korrigiertes Graustufenbild einlesen und aufspalten
 subimgs = subimg(npframe)

 for img in subimgs:		# Mittelwert und Standartabweichung des Bildes ausgeben
     avg, std = avg_stdabw(img)1
     print(str(avg) + " & " + str(std))
\end{lstlisting}