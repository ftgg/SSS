\begin{lstlisting}[style=PYTHON,frame=single,
 caption=Zum einlesen der Bilder von Kamera oder Verzeichniss
 captionpos=b,
 label=lst:CVREAD] 
 import cv2
 import sys
 import numpy as np
 
 def bildoeffnen(filename):					# ließt ein Bild von Verzeichniss als Matrix ein
     pic = cv2.imread(filename)
     pic = cv2.cvtColor(pic, cv2.COLOR_BGR2GRAY)
     return pic
 
 def einlesen(filename):					# nimmt ein Bild von Kamera als Matrix auf
     cap = cv2.VideoCapture(0)
     if (not cap.isOpened()):				# fehlerbehandlung, falls keine Kamera gefunden
         print("fehlgeschlagen")
         sys.exit(-1)
     while(True):							# dauerhaft aktualisiertes Bild auf Monitor ausgeben
         ret, frame = cap.read()
         gray = cv2.cvtColor(frame, cv2.COLOR_BGR2GRAY)	# umwandlung in ein Graubild
         cv2.imshow("frame", gray)			# anzeigen des Bildes
         if(cv2.waitKey(1) & 0xFF == ord('q')):
             cv2.imwrite(filename, gray)	# bei betätigung der Taste 'q' wird das Bild gespeichert
             break;          
     cap.release()
     cv2.destroyAllWindows()
 
 def contrastMax(pic):					# sorgt für einen maximalen Kontrast des übergebenen Bildes
     newPic = np.zeros((pic.shape[0],  pic.shape[1]), dtype=np.uint8)
     minVal = np.min(pic)				# minimalwert, später 255
	 maxVal = np.max(pic)				# maximalwert, später 0
     for y in range(0,pic.shape[0]):
         for x in range(0,pic.shape[1]):
             if (maxVal-minVal) != 0:	# division durch 0 vermeiden
                 newPic[y][x]= (float(pic[y][x]- minVal)*(255.0/(maxVal-minVal)))
             else:
                 newPic[y][x]= 0		# falls maxval == minval wird das Bild schwarz
     return newPic
 
 #for i in range(1,11):					# beispielschleife zum Einlesen der 10 dunkelbilder
 #    einlesen("../Bilder/dunkelbild" + str(i) + ".png")
 
 def print_inf():						# gibt Informationen über die Kameraeinstellung aus
     cap = cv2.VideoCapture(0)
     print("frame width:   " + str(cap.get(3)))
     print("frame height:  " + str(cap.get(4)))
     print("----------------------------------")
     print("brightness:    " + str(cap.get(10)))
     print("contrast:      " + str(cap.get(11)))
     print("saturation:    " + str(cap.get(12)))
     print("----------------------------------")
     print("gain:          " + str(cap.get(14)))
     print("exposure:      " + str(cap.get(15)))
     print("----------------------------------")
     print("white balance: " + str(cap.get(17)))
     cap.release()

 \end{lstlisting}