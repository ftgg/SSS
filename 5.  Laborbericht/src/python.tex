\begin{lstlisting}[style=PYTHON,frame=single,
 caption=Zum einlesen der Mundharmonika Schwingung
 captionpos=b,
 label=lst:read] 
 import numpy as np
 from TekTDS2000 import *
 
 scope = TekTDS2000()
 
 input()
 scope.saveCsv(filename='mic1000Hz.csv', ch=2)
 scope.saveCsv(filename='sig1000Hz.csv', ch=1)
 scope.plot(filename='mic1000Hz.png', ch=2)
 scope.plot(filename='sig1000Hz.png', ch=1)
 scope.getSamplingInterval()
 scope.getRecordLength()
 # ausgabe bei nachmessung:
 #TekTDS2000: Sampling interval: 2e-06 s
 #TekTDS2000: Record length: 2500
 \end{lstlisting}
 
  
  \begin{lstlisting}[style=PYTHON,frame=single,
   caption=Zum umwandeln der Mundharmonika Schwingung
   captionpos=b,
   label=lst:v1d]
   import numpy as np
   import matplotlib.pyplot as plt
   
   #Periodendauer
   def periodendauer():
       input_data = np.genfromtxt('tone.csv', delimiter=',')
       my_data = input_data[:,1:2];
       
       plt.figure(figsize=(10, 5), dpi=800)
       plt.xticks(range(0,271,10))
       plt.plot(my_data[0:270], label='Signal')
       plt.plot((44, 44), (-0.008, 0.008), 'k-', color='red')      # anfang Periode
       plt.plot((153, 153), (-0.008, 0.008), 'k-', color='red')    # ende Periode
       plt.ylabel("Amplitude")
       plt.xlabel("Abtastzeitpunkt")
       plt.text(57, -0.0065, '$periodendauer = 109 * 10^{-5} s$', fontsize=14)
       plt.legend()
       plt.show()
   
   
   # Fouriertransformation
   def mach_fft(data):
       fft = np.abs(np.fft.fft(data, axis=0))
       plt.figure(figsize=(10, 5), dpi=800)
       plt.xlabel("Frequenz [$Hz$]")
       plt.ylabel("Amplitude")
       plt.plot(range(0,16000,40), fft[:400], label="Fourierdom�ne")   # gek�rzt bei 400, rest uninteressant
       plt.plot((910, 910), (3, 0), 'k-', color='red',label="Grundschwingung = 910 Hz") # Grundfrequenz
       #plt.plot((1820, 1820), (3, 0), 'k-', color='red') # 1 Harmonische
       #plt.plot((2730, 2730), (3, 0), 'k-', color='red') # 2 Harmonische
       plt.legend()
       plt.show()
   \end{lstlisting}
 
 \begin{lstlisting}[style=PYTHON,frame=single,
  caption=Erstellen der Grafiken
  captionpos=b,
  label=lst:plot]
  import numpy as np
  import matplotlib.pyplot as plt
  
  # vergleich der Eingangsspannung zum mikrophon Signal
  def ein_ausgang():
      input_sig = np.genfromtxt('sig1000Hz.csv', delimiter=',')
      input_mic = np.genfromtxt('mic1000Hz.csv', delimiter=',')    
      
      plt.figure(figsize=(8, 5), dpi=800)
      plt.xticks(range(0,2501,250))
      plt.ylabel("Amplitude")
      plt.xlabel("Abtastzeitpunkt")
      resize_mic = input_mic[:,1:2] * 40  # mal 40, da mikrophonsignal etwa 40 mal schw�cher ist
      plt.plot(input_sig[:,1:2], color='y', label='1kHz Eingang')
      plt.plot(resize_mic, color='b', label='Ausgang')
      plt.plot((845, 845), (-1, 1), '-', color='red',label='Phasenverschiebung')
      plt.plot((940, 940), (-1, 1), '-', color='red')
      plt.legend()
      plt.show()
  
  
  # Bode-Diagramm f�r Amplitude
  def amplitude(datak, datag, x):
      datak = datak / 1500   # normierung
      datak = np.log10(np.abs(datak))
      datak = datak * 20    # dB
      datag = datag / 1500   # normierung
      datag = np.log10(np.abs(datag))
      datag = datag * 20    # dB
      plt.figure(figsize=(8, 5), dpi=800)   
      plt.semilogx(x, datak, label="kleiner Lautsprecher")
      plt.semilogx(x, datag, label="gro�er Lautsprecher")
      plt.xlabel("Frequenz")
      plt.ylabel("Amplitudengang [$dB$]")
      plt.legend()
      plt.show()
  
  
  # Bode-Diagramm f�r Phasengang
  def phase(datak, datag, x):
      plt.figure(figsize=(8, 5), dpi=800)   
      plt.semilogx(x, datak, label="kleiner Lautsprecher")
      plt.semilogx(x, datag, label="gro�er Lautsprecher")
      plt.xlabel("Frequenz")
      plt.ylabel("Phasengang")
      plt.legend()
      plt.show()
      
  # Beispielaufruf:
  # x = alle Messwerte, amplitude_gross = alle gemessenen Amplitudenwerte
  # zur jeweiligen frequenz, siehe Tabelle im Bericht
  
  #x = np.array([100,200,300,400,500,700,850,1000,1200,1500,1700,2000,3000,4000,5000,6000,10000])
  
  #amplitude_gross = np.array([50.2,134.0,95.2,60.4,49.2,
  #                   39.6,43.2,42.0,35.2,33.6,
  #                   30.8,32.0,37.6,46.0,26.8,
  #                   19.4,15.4])
  
  #amplitude_klein = np.array([6.6,10.4,15.0,23.8,34.6,
  #                   70.0,68.0,50.0,40.0,52.8,
  #                   53.6,51.2,50.4,52.0,12.8,
  #                   8.2,19.4])
                 
  #amplitude(amplitude_klein, amplitude_gross, x)
  
 \end{lstlisting}
 