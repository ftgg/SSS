\begin{lstlisting}[style=PYTHON,frame=single,
 caption=Zum einlesen der Mundharmonika Schwingung
 captionpos=b,
 label=lst:read] 
import pyaudio
import numpy as np
import matplotlib.pyplot as plt
import scipy.signal as wim
import scipy.stats

FORMAT = pyaudio.paInt16
SAMPLEFREQ = 44100
FRAMESIZE = 1024
NOFFRAMES = 220
TRIGGER = 400

def trigger(signal):  
    for i in range(0,len(signal)):
        if signal[i] > TRIGGER:
            signal = signal[i:]
            break
    restlength = min(len(signal),44100)
    zeros = np.array(np.zeros(44100))
    for i in range(0, restlength):
        zeros[i] = signal[i]
    for i in range(0,44100):
        if zeros[44099-i] > TRIGGER/2:
            break
        zeros[44099-i] = 0
    return zeros

def record():
    p = pyaudio.PyAudio()
    print("running")
    stream = p.open(format=FORMAT, channels=1, rate=SAMPLEFREQ, input=True)
    data = stream.read(SAMPLEFREQ*2)
    decoded = np.fromstring(data, 'Int16')
    stream.stop_stream()
    stream.close()
    p.terminate()
    print("done")
    decoded = trigger(decoded)
    plt.plot(decoded)
    plt.show()

    file = "fab.csv"
    np.savetxt(file,decoded,delimiter=',')
    return decoded

def fft_window(signal):
    gauss = wim.gaussian(512,std=512/4)
    fourier = np.zeros(256)
    tmp = np.zeros(512)
    for i in range(0,44100 - 512, 256):
        for a in range(0,512):
            tmp[a] = gauss[a] * signal[i + a]
        fourier = fourier + np.abs(np.fft.fft(tmp, axis=0)[:256])
    fourier = (fourier / 171)/450
    return fourier

def fft_std(data, filename=''):
    fft = np.abs(np.fft.fft(data, axis=0))
    plt.figure(figsize=(800/75, 600/75), dpi=75)
    plt.plot(fft[:5000]/len(fft))  
    plt.autoscale(enable=True,axis = "x",tight = True)
    plt.xlabel('Frequency [$Hz$]')
    plt.ylabel('$\\left| Y \\left( f \\right ) \\right|$')
    if filename is not '':
        plt.savefig(filename+'.png', dpi=75)
        plt.close()
    plt.show()
    return 
    
def ref_spektren(filename):
    spektrum = np.zeros(256)
    for id in range(0,5):
        data = np.genfromtxt("{}_{}.csv".format(filename, id), delimiter=',')
        data = fft_window(data) #spektrum erzeugt
        spektrum = spektrum + data;
    spektrum = spektrum / 5
    return spektrum

def get_freq(M):
    dt = 1 / SAMPLEFREQ
    freq = np.zeros(M)    
    for i in range(0,M):
        freq[i] = i / (M*dt)
    return freq

def save_spekt(filename):
    plt.figure(figsize=(800/75, 600/75), dpi=75)
    ref = ref_spektren(filename)
    plt.xlabel('Frequency [$Hz$]')
    plt.ylabel('$\\left| Y \\left( f \\right ) \\right|$')
    plt.plot(get_freq(50)[:25],ref[:25],c='b',lw=1)
    plt.show()
    np.savetxt(filename,ref,delimiter=',')
    
def korellation(spekA, spekB):
    return scipy.stats.pearsonr(spekA, spekB)    

def spracherkenner(wort):
    hoch = np.genfromtxt("Aufnahmen_Referenz_Hoch", delimiter=',')
    tief = np.genfromtxt("Aufnahmen_Referenz_Tief", delimiter=',')
    links = np.genfromtxt("Aufnahmen_Referenz_Links", delimiter=',')
    rechts = np.genfromtxt("Aufnahmen_Referenz_Rechts", delimiter=',')
    wortschatz = { "Hoch" : hoch, "Tief" : tief, "Links" : links, "Rechts": rechts}
    
    wort = fft_window(wort)
    max = 0;
    finalkey = "nicht gefunden"
    
    for key in wortschatz:
        korrel = korellation(wort,wortschatz[key]) 
        if(korrel[0] > max):
            max = korrel[0]
            finalkey = key
    
    
    if(max < 0.9):
        finalkey = "nicht gefunden"
    else: 
        print("Wort: {}, Korrellation: {}".format(finalkey,max))
    return finalkey
    
def all_data():
    number = 0
    success = 0
    for sprecher in ["A","B"]:
        for wort in ["Hoch","Tief","Links","Rechts"]:
            for i in [0,1,2,3,4]:
                number = number + 1
                print("Datei : Aufnahmen_Test_{}_{}_{}.csv:".format(sprecher, wort, i))
                finalkey = spracherkenner(np.genfromtxt("Aufnahmen_Test_{}_{}_{}.csv".format(sprecher, wort, i), delimiter=','))
                if(finalkey == wort):
                    print("Happy") 
                    success = success+1
                    
    print("Erfolgreich festgestellt sind {} von {} worten, {}%".format(success,number,(success / number) * 100))
    return     
all_data()

\end{lstlisting}